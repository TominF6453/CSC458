\documentclass[12pt]{article}
\usepackage{fullpage,amsmath,amssymb,amsthm}
\usepackage{tikz}
\usepackage{hyperref}
\hypersetup{pdfpagemode=FullScreen}
\usepackage{mathtools}
\usepackage{qtree}
\DeclarePairedDelimiter{\ceil}{\lceil}{\rceil}
\DeclarePairedDelimiter{\floor}{\lfloor}{\rfloor}

\title{CSC458 Problem Set \#1\\}
\author{Filip Tomin\\
\normalsize{1001329984}}

\date{}
\begin{document}
\maketitle
\begin{enumerate}
\item %Chapter 1: 13
\begin{enumerate}
\item % (A)
The minimum RTT would be the RTT for a single bit, which would just be $(3.85\times 10^8_m) / (3\times 10^8_{m/s}) = 1.283_s\times 2 = 2.57_s$.
\item % (B)
Bandwidth-delay product is $(1.0\times 10^9)\times 2.57 = 2.57\times 10^9$ or 2.57Gb.
\item % (C)
The bandwidth-delay product is extremely large. Assuming the wire is being used fully, there is a lot of data that the wire can hold unacknowledged by the destination. Slow response time is also a problem for protocols such as TCP.
\item % (D)
Assuming no errors and the image is sent as one packet, the delivery time will be the transmission time + propagation delay.\\
Transmission time: $(2.5\times 10^7_B)\times 8 / (1.0\times 10^9_{b/s}) = 0.2_s$\\
Propagation delay is just $1.28_s$, half of the RTT.\\
If we assume the request to be extremely small, such as 1 bit, then the total time it would take would be $0.2 + 1.28 + 1.28 = 2.76$ seconds.
\end{enumerate}
\item %Chapter 1: 16
\begin{enumerate}
\item % (A)
Latency = $\sum_i (Prop_i + M/R_i)$\\
Latency = $2((1.0\times 10^{-5}) + (1.2\times 10^4)/(1.0\times 10^8)) = 0.00026s$ or $260\mu s$.
\item % (B)
Latency = $4((1.0\times 10^{-5}) + (1.2\times 10^4)/(1.0\times 10^8)) = 0.00052s$ or $520\mu s$.
\item % (C)
Latency = $M/R_{min} + \sum_i Prop_i$\\
Latency = $(1.2\times 10^4)/(1.0\times 10^8) + 2(1.0\times 10^{-5}) = 0.00014s$ or $140\mu s$.
\end{enumerate}
\item %Chapter 1: 19
\begin{enumerate}
\item % (A)
RTT is the one-way delay times 2, so RTT = $2\times 10\mu s = 20\mu s$\\
Bandwidth-delay product is $(1.0\times 10^8)\times (2.0\times 10^{-5}) =  2000$b or 2kb.
\item % (B)
RTT = $520\mu s$ from \#16.\\
Bandwidth-delay product = $(1.0\times 10^8)\times (5.2\times 10^{-4}) = 52000$b or 52kb.
\item % (C)
RTT = $2\times 50$ms$ = 100$ms.
Bandwidth-delay product = $(1.5\times 10^6)\times (0.1) = 1.5\times 10^5$ or 150kb.
\item % (D)
RTT = $4\times ((3.59\times 10^7)/(3\times 10^8)) = 479$ms.
Bandwidth-delay product = $(1.5\times 10^6)\times (0.479) = 7.185\times 10^5$ or 718kb.
\end{enumerate}
\item %Chapter 1: 26
\begin{enumerate}
\item % (A)
Total bytes per second $= 640\times 480\times 3\times 30 = 27.7$MB.\\
Bandwidth must be $\geq 27.7$MB/s or 221.6Mbps.
\item % (B)
Total bytes $= 160\times 120\times 1\times 5 = 96$kB.\\
Bandwidth must be $\geq 96$kB/s or 768kbps.
\item % (C)
Total bytes $= (6.5\times 10^8)/4500 = 144$kB.\\
Bandwidth must be $\geq 144$kB/s or 1.16Mbps.
\item % (D)
Since it's black and white, assume each pixel requires only one bit.\\
Total bits $= 8\times 72\times 10\times 72 = 414720$ bits.\\
$414720/14400 = 28.8$s.
\end{enumerate}
\item %Chapter 2: 17
The Internet checksum is usually calculated by taking the ones complement sum in 16-bit units. This method is equivalent since the 32-bit sum is the same as two side-by-side 16-bit sums, even maintaining the overflow rules. The 32-bit sum is converted into the 16-bit sum next. The only problem is this 16-bit value has been ones complemented 2 times, so it is inverted from what it should be. Thus, we take the ones complement of the result again, to produce the checksum value.\\
32-bit:
\begin{center}
\begin{tabular}{r l}
0001101000010101 & 0000000100110000\\
+ 0000000000011101 & 0000001001010010\\\hline
0001101000110010 & 0000001110000010\\\hline
1's comp - 1110010111001101 & 1111110001111101
\end{tabular}\\
\begin{tabular}{r}
1110010111001101\\
+ 1111110001111101\\\hline
1110001001001011\\\hline
1's comp - 0001110110110100\\\hline
1's comp - \textbf{1110001001001011}
\end{tabular}
\end{center}
16-bit:
\begin{center}
\begin{tabular}{r}
0001101000010101\\
0000000100110000\\
0000000000011101\\
+ 0000001001010010\\\hline
0001110110110100\\\hline
1's comp - \textbf{1110001001001011}
\end{tabular}
\end{center}
As shown above, both methods ended up with the same 16-bit checksum value.\pagebreak
\item %Chapter 2: 18
\begin{enumerate}
\item % (A)
$M(x) = 11100011$, $G(x) = 1001$, $x^rM(x) = 11100011000$
\begin{verbatim}
     _____________
1001 | 11100011000
       1001
         1100
         1001
           1011
           1001
             1000
             1001
                10 = Remainder
\end{verbatim}
$T(x) = 11100011010$, this is the message that should be transmitted.
\item % (B)
The received bits would be 01100011010. Then:
\begin{verbatim}
     _____________
1001 | 01100011010
       1001
        1110
        1001
          1101
          1001
            1010
            1001
               110 = Remainder
\end{verbatim}
Since the remainder does not equal 0, the receiver knows there has been an error.
\end{enumerate}
\item %Chapter 2: 46
In this scenario, the successful transmissions occur in the order of C, B, A, the attempted transmissions occur in the order of A, B, C, and there are at least 4 collisions.
\begin{verbatim}
D----------| C----------| B----------| A----------|
  ^    ^     ^     ^      ^     ^      ^
  A    B     C     A      B     A      A
  |    |           |            |
  \____\___________\____________\___-- Collisions.
\end{verbatim}\pagebreak
\item %Chapter 3: 3
Based on the graph:\\
\begin{tabular}{c c c}
A:
\begin{tabular}{c|c}
Destination & Next\\\hline
B & C\\
C & C\\
D & C\\
E & C\\
F & C\\
\end{tabular} &
B:
\begin{tabular}{c|c}
Destination & Next\\\hline
A & E\\
C & E\\
D & E\\
E & E\\
F & E\\
\end{tabular} &
C:
\begin{tabular}{c|c}
Destination & Next\\\hline
A & A\\
B & E\\
D & E\\
E & E\\
F & F\\
\end{tabular}\\\\
D:
\begin{tabular}{c|c}
Destination & Next\\\hline
A & E\\
B & E\\
C & E\\
E & E\\
F & E\\
\end{tabular} &
E:
\begin{tabular}{c|c}
Destination & Next\\\hline
A & C\\
B & B\\
C & C\\
D & D\\
F & C\\
\end{tabular} &
F:
\begin{tabular}{c|c}
Destination & Next\\\hline
A & C\\
B & C\\
C & C\\
D & C\\
E & C\\
\end{tabular}
\end{tabular}
\item %Chapter 3: 15 % A>C, C>A, D>C
For this answer, A-$>$B means that the bridge has knowledge that to get to A, it must send the packet to B.\\
\begin{tabular}{c}
A sends to C\\
\begin{tabular}{c|c|c|c}
B1 & B2 & B3 & B4\\\hline
A-$>$A & A-$>$B1 & A-$>$B2 & A-$>$B2\\
\end{tabular}\\\\
C sends to A\\
\begin{tabular}{c|c|c|c}
B1 & B2 & B3 & B4\\\hline
A-$>$A & A-$>$B1 & A-$>$B2 & A-$>$B2\\
C-$>$B2 & C-$>$B3 & C-$>$C & \\
\end{tabular}\\\\
D sends to C \\
\begin{tabular}{c|c|c|c}
B1 & B2 & B3 & B4\\\hline
A-$>$A & A-$>$B1 & A-$>$B2 & A-$>$B2\\
C-$>$B2 & C-$>$B3 & C-$>$C & \\
& D-$>$B4 & D-$>$B2 & D-$>$D\\
\end{tabular}
\end{tabular}
\item %Chapter 3: 19
\begin{enumerate}
\item % (A)
The packet will go to both B1 and B2, which will record M-$>$M. Then they will send the packet to L and the other bridge. Since neither of the bridges has an entry for L in their table, they will continue to send the packet to each other in a loop, while also giving L the packet multiple times. This will also replace the M-$>$M records with M-$>$B1 or B2.
\item % (B)
Let's assume that the packet from (a) is already circling clockwise around the bridges. When L sends its packet, B1 has just obtained the packet from M. B1 believes M is at the bottom port and B2 believes M is at the top port. The packet from L which hits B1 is sent to B2 from the bottom and then B1 again from the top, a counter-clockwise loop. Now, a packet from M is moving clockwise around the loop and a packet from L is moving counter-clockwise around the loop.
\end{enumerate}
\end{enumerate}
\end{document}